%xelatex
\documentclass{article}
\usepackage[top=1cm,bottom=2cm]{geometry}
\usepackage{titling}
\usepackage{titlesec}
\usepackage{hyperref}
\usepackage{polyglossia}
\newfontfamily\arabicfont[Script=Arabic]{Amiri}
%\newfontfamily\englishfont{DejaVu Serif}
\setmainlanguage{persian}
\setotherlanguage{english}

\pagestyle{empty}
\title{راهنمای ناوبری سامانه}
\author{بنیان‌گذار: لوک اسمیت}
\date{ویرایش: مجتبی وحیدی نسب}

\newcommand{\s}[1]{\textenglish{Mod+#1}}
\renewcommand{\maketitle}{
	\begin{center}
		{\huge\thetitle}

		{\footnote{\raggedright\textenglish{Luke Smith}}{\theauthor}\hspace{2cm}\thedate}
	\end{center}
}

\titleformat{\section}{\Large\bfseries}{}{0cm}{}
\titleformat{\subsection}{\large\bfseries}{}{0cm}{}

\begin{document}
\maketitle
این یک سامانه شخصی‌سازی شده از \footnote{\raggedright\textenglish{LARBS: Luke's Auto Rice Bootstrap Script}}{LARBS} می‌باشد. با کلیدهای \textenglish{h j k l} جابجا شوید. با کلید s درازای سند، نمایشگرتان را پوشش می‌دهد و با کلید a پهنای سند، پهنای نمایشگرتان را پوشش می‌دهد.
\section{ارزش‌‌ها}
\begin{itemize}
	\item اقتصادی: همه برنامه‌ها ساده هستند و منابع کمی مصرف می‌کنند و بسیار گسترش‌پذیرند.
	\item تخته‌کلید: همه برنامه‌ها با تخته‌کلید مهار می‌شوند و نیازی نیست انگشتان از روی تخته‌کلید برداشته شوند.
\end{itemize}
\section{تغییرات کلی تخته‌کلید}
\begin{itemize}
	\item \textenglish{CapsLock}: این یک کلید بی‌مصرف در مکان مهمی است. این کلید نقش Esc را دارد.
	\item \textenglish{Menu}: این نقش کلید Windows اضافی در سمت راست تخته‌کلید را دارد.
\end{itemize}
\section{نوار وضعیت}
\indent\indent{در سمت چپ شماره برچسب را می‌بینید. در سمت راست موارد دیگر مانند تاریخ شمسی(جلالی) و میلادی و صدا و ... . هر یک از این موراد در مسیر \textenglish{\textasciitilde/.local/bin/statusbar/} قرار دارند و می‌توانید آنها را تغییر دهید. همچنین می‌توانید با نگهداشتن کلید Shift و زدن روی هرکدام، آن را تغییر دهید.}

برنامه dwmblocks سرپرست نمایش و بروزرسانی نواروضعیت را دارد. می‌توانید این برنامه را از مسیر \textenglish{\textasciitilde/.local/src/dwmblocks/} تغییر دهید.
\section{میانبرها}
\indent\indent{مدیر پنجره dwm سرپرست اجرای میانبرها می‌باشد. می‌تواند در مسیر \textenglish{\textasciitilde/.local/src/dwm/} میانبرها را تغییر دهید. به بزرگ یا کوچک بودن حروف میانبر توجه کنید.}
\begin{itemize}
		 \item میانبر \s{Enter} فراخوانی پایانه
		 \item میانبر \s{q} بستن پنجره
		 \item میانبر \s{d} فهرست برنامه‌ها برای اجرا
		 \item میانبر \s{j/k} مابین پنجره‌ها جابجا شوید
		 \item میانبر \s{Space} پنجره فعال را اصلی می‌کند یا با دومی جابجا می‌کند
		 \item میانبر \s{z/x} شکاف بین پنجره‌ها را زیاد یا کم می‌کند
		 \item میانبر \s{a} شکاف را فعال یا غیرفعال می‌کند
		 \item میانبر \s{A} شکاف را به پیشفرض برمی‌گرداند
		 \item میانبر \s{Shift+Space} پنجره را شناور می‌کند. پنجره‌های شناور با \s{left/right click} بابجا یا بزرگ و کوچک می‌شوند
		 \item میانبر \s{s} پنجره را چسبان می‌کند (همراه شما می‌ماند)
		 \item میانبر \s{b} نوار وضعیت را پنهان و آشکار می‌کند
\end{itemize}
\subsection{چیدمان}
\begin{itemize}
	\item \s{t} چیدمان
	\item \s{T} چیدمان
	\item \s{y} چیدمان
	\item \s{Y} چیدمان
	\item \s{u} چیدمان
	\item \s{U} چیدمان
	\item \s{i} چیدمان
	\item \s{I} چیدمان
	\item \s{f} چیدمان
	\item \s{F} چیدمان
	\item \s{o/O} چیدمان
\end{itemize}
\subsection{برنامگان}
\begin{itemize}
	\item \s{w} مرورگر
	\item \s{W} وای‌فای
	\item \s{e} برنامه neomutt برای رایانامه
	\item \s{E} برنامه abook برای مخاطبین
	\item \s{r} مدیرپرونده lf را باز می‌کند
	\item \s{R} سرپرست کار htop را بازمی‌کند
	\item \s{n} برنامه vimwiki برای یادداشت برداری
	\item \s{N} برنامه newsboat برای اخبار
	\item \s{'} ماشین حساب bc را باز و بسته می‌کند
	\item \s{Shift+Enter} پایانه شناور
\end{itemize}
\subsection{برچسب‌ها}
\begin{itemize}
	\item \s{(\textpersian{شماره})} به میزکار دیگری می‌روید
	\item \s{Shift+(\textpersian{شماره})} پنجره را به میز کار دیگری می‌فرستد
	\item \s{\textpersian{راست/چپ}} به نمایشگر دیگری می‌روید
	\item \s{Shift+\textpersian{راست/چپ}} پنجره را به نمایشگر دیگری می‌فرستد
\end{itemize}
\subsection{سامانه}
\begin{itemize}
	\item \s{Q} انتخاب کنید تا نمایشگر را قفل کنید، خارج شوید، خاموش کنید یا خاموش و روشن کنید
	\item \s{Backspace} انتخاب کنید تا نمایشگر را قفل کنید، خارج شوید، خاموش کنید یا خاموش و روشن کنید
	\item \s{F1} این سند را نمایش می‌دهد
	\item \s{F2} ویدیو آموزشی درمورد سامانه ببینید
	\item \s{F3} نمایشگرتان را انتخاب کنید
	\item \s{F4} برنامه pulsemixer برای مهار آوا
	\item \s{F6} ---
	\item \s{F7} ---
	\item \s{F9} سوار کردن یک حافظه خارجی
	\item \s{F10} پیاده کردن یک حافظه خارجی
	\item \s{F11} نمایش دوربین
	\item \s{F12} ---
	\item \s{`} جایگذاری شکلک
	\item \s{Insert} جایگذاری نشانی اینترنتی. از \textenglish{\textasciitilde/.local/share/larbs/snippets/urls} تغییر بدید
	\item \s{Shift+Insert} جایگذاری بریده‌های دیگر بجر نشانی اینترنتی. بریده خودتان را در \textenglish{\textasciitilde/.local/share/larbs/snippets/} ذخیره کنید. همچنین کارکرد این میانبر را با تغییر \textenglish{\textasciitilde/.local/bin/snippet} تغییر دهید
\end{itemize}
\subsection{آوا}
\begin{itemize}
	\item \s{m} پخش کننده موسیقی ncmpcpp را باز می‌کند
	\item \s{M} بی‌صدا و باصدا کردن
	\item \s{p} قطع و پخش موسیقی
	\item \s{P} قطع و پخش موسیقی
	\item \s{-} صدا را کم می‌کند. کلید Shift را نگهدارید تا بیشتر کم کند
	\item \s{+} صدا را زیاد می‌کند. کلید Shift را نگه دارید تا بیشتر زیاد کند
	\item \s{.} موسیقی بعدی در قهرست را پخش می‌کند
	\item \s{,} موسیقی قبلی در فهرست را پخش می‌کند
	\item \s{>} فعال و غیرفعال کردن تکرار فهرست‌پخش
	\item \s{<} بازآغاز موسیقی کنونی
	\item \s{\lbrack} ده ثانیه عقبگرد موسیقی
	\item \s{\rbrack} ده ثانیه جلوگرد موسیقی
\end{itemize}
\subsection{ضبط}
\begin{itemize}
		 \item کلید PrintScreen عکس‌ازنمایشگر می‌گیرد
		 \item میانبر \textenglish{ShiftPrintScreen} گزینه‌های بیشتر برای عکس‌ازمنایشگر
		 \item \s{PrintScreen} گزینه برای ضبط نمایشگر
		 \item میانبر \s{Delete} ضبط کردن را پایان می‌دهد
		 \item میانبر \s{Ctrl+PrintScreen} کلیدهای فشرده شده را نمایش می‌دهد که برای ضبط نمایشگر بدرد می‌خورد
\end{itemize}
\appendix
\section{پیوست}
\raggedright\href{https://lukesmith.xyz}{https://lukesmith.xyz}\bigskip

\raggedright\href{https://larbs.xyz}{https://larbs.xyz}\bigskip

\raggedright\href{https://github.com/lukesmithxyz/LARBS}{https://github.com/lukesmithxyz/LARBS}\bigskip

\raggedright\href{https://github.com/mojtabavahidinasab/LARBS}{https://github.com/mojtabavahidinasab/LARBS}\bigskip

\raggedright\href{https://github.com/lukesmithxyz/voidrice}{https://github.com/lukesmithxyz/voidrice}\bigskip

\raggedright\href{https://github.com/mojtabavahidinasab/voidrice}{https://github.com/mojtabavahidinasab/voidrice}\bigskip

\raggedright\href{https://github.com/lukesmithxyz/dwm}{https://github.com/lukesmithxyz/dwm}\bigskip

\raggedright\href{https://github.com/mojtabavahidinasab/dwm}{https://github.com/mojtabavahidinasab/dwm}\bigskip

\raggedright\href{https://github.com/lukesmithxyz/dwmblocks}{https://github.com/lukesmithxyz/dwmblocks}\bigskip

\raggedright\href{https://github.com/mojtabavahidinasab/dwmblocks}{https://github.com/mojtabavahidinasab/dwmblocks}\bigskip

\raggedright\href{https://github.com/lukesmithxyz/st}{https://github.com/lukesmithxyz/st}\bigskip

\raggedright\href{https://github.com/mojtabavahidinasab/st}{https://github.com/mojtabavahidinasab/st}\bigskip

\end{document}
